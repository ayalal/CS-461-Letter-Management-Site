\section{Project Documentation}

\subsection{Project Overview/Implementation}
\noindent The main operation of our project is to create a work flow between professors and students of Oregon State University when handling letters of recommendation. In order to connect the university to our application we integrated Oregon State's CAS login system so that professors and students can utilize there already existing accounts on our application. The application then provides two unique interfaces dependent on if the account is a student or professor. Students are provided a interface where they can upload files, search for a desirable letter candidate, and make requests through that previous search. Professors are able to set preferences for students to view so that the student is aware of what documentation to upload for that professor. Professors can also handle all letter requests from students via a accept/deny system which based on the decision will provide the necessary follow up. Should the recommendation be accepted the professor will have access to the students file which will allow the professor to begin working on the letter of recommendation. A schedule will also be created between both parties with a predetermined deadline for the letter and a description on the request. Should a request be denied the student will be sent an email stating that the specific professor has denied their request. After completing a letter a professor can then submit that letter to the specific student who can then use it for future use/reference.

\subsection{Software Installation/Operation}
\noindent Handling our project will be difficult for those who are outside the scope of our project since it relies on an AWS account between our group and the client. Permissions to this account can not be handed out either since the application uses sensitive information. However, they is a way to use the application from a local instance but with the caveat that not all features will be present and/or work correctly. In order to run the web application locally there are a few software requirements that are necessary to handle. First, the user will need to install a version of Python 3 and the python install pip3 to use the commands needed to run our application. In order to aid our users we created a install script that will handle the previous stated software installation as well as handle all package dependencies utilized with our application as python allows for the setup of a requirements text file which can be used as a dump for all package arguments for future installation on other systems. After completing the required installation the script will then run the application locally for the user via the command 'python3 manage.py runserver'. Should the user desire to stop the server they can 'ctrl-C' to stop then if they want to start the application back up simply run the 'python3 manage.py runserver' argument again to restart. In terms of a user guide we will have one listed on our Github repository for more in depth usage of our web application.    