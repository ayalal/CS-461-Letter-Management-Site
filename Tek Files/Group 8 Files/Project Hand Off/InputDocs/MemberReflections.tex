\section{Team Member Reflections}
\subsection{Lorenzo Ayala}\\
\subsubsection{What technical information did you learn?}
\noindent Over the duration of this project I learned just how extensive a web application can become even from the perspective of adding a micro-service to an existing organization. I also learned one of the most utilized frameworks.\\

\subsubsection{What non-technical information did you learn?}
\noindent In terms of non-technical aspects I learned what can be expected from projects in terms of documentation and preparing the necessary paperwork which can provide a strong analysis on your product while also being a guide on how the product operates. Also time management is key when handling large projects such as this.\\ 

\subsubsection{What have you learned about project work?}
\noindent I learned that delegation of tasks is essential in developing a product since it is always easier to conquer an issue broken up in smaller problems to solve rather than trying to solve them by yourself head on. In relation to delegating tasks, always match up tasks with team member strengths for efficiency and effectiveness of solutions.\\ 

\subsubsection{What have you learned about project management?}
\noindent Similar to the non-technical information, time management is key when it comes to large projects such as this one. Projects for senior capstone are ones that can't be done the weekend before it takes determination, efficiency, and work ethic to create a product to be proud of. Also, as I said before delegate tasks among your group to match strengths and avoid weaknesses to be an efficient group that is enjoying working on the project. I believe that these reasons are why we had a good time working on our project. Finally, I would like to add that do not be afraid to ask for help as there are so many helpful individuals at Oregon State University who could help you solve an issue in quick amount of time (rather than waste time trying to solve it yourself).\\

\subsubsection{What have you learned about working in teams?}
\noindent Having previously work at Cambia Health Solutions I was accustomed to working with teams on an industry level so I was able to sharpen the skills I learned from them. Following that statement, always go in with an open mind it is great seeing the various solutions and implementations your teammates come up with and can provide you with an alternative perspective you may have missed otherwise. Communication with each other is essential and can determine how development can go as well as other aspects of the project. Lastly, have fun with one another it makes the entire project enjoyable and memorable!\\

\subsubsection{If you could do it all over, what would you do differently?}
\noindent Probably have more courage to ask for help when I needed it while working on the project. I tend to shy away from help since it makes me feel as if I failed to implement some aspect of the project. Erasing this type of mindset was huge since asking for help from others provided quick solutions while giving me learning opportunities from individuals who have a wide array of knowledge in computer science. Also, if you can, start your project as early as possible. While we did not wait last minute to do our project we didn't start as soon as possible and that time could have been used to make something even more spectacular. Finally once again have fun this is a great learning experience and a way to introduce yourself to professionals within/outside Oregon State University.\\

\subsection{Johnathan Lee}\\

\subsubsection{What technical information did you learn?}
\noindent Some technical information I learned over the course of the year was learning how to code in Django environment and utilizing its database. I got to experience making a website from the bottom up and develop it from scratch. It was a nice experience learning how to make a website coding in Python because it is different from coding in C style languages.\\

\subsubsection{What non-technical information did you learn?}
\noindent Some of the non-technical information that I learned about included keeping track of documentation and recording daily progress on a huge project. It really puts in perspective what was done over the course of the year. \\ 

\subsubsection{What have you learned about project work?}
\noindent I have learned that big projects can have a lot of work involved and that teammates need to communicate clearly to properly cover the entire scope of the project. Overlapping or missing parts from the project could cause confusion within the team and hinder progress.\\ 

\subsubsection{What have you learned about project management?}
\noindent Some things that I learned about project management is that it’s a lot easier when there is a team leader coordinating the project. Things can quickly fall apart when there’s a lack of communication between team members.\\

\subsubsection{What have you learned about working in teams?}
\noindent I learned that communication can be important in working in teams and that assumptions can never be made about anything so everything needs to be clearly expressed before confirmation.\\

\subsubsection{If you could do it all over, what would you do differently?}
\noindent If I could do it all over again, there is not too much that I would like to change. However, some things I would like to try is offering my teammates some help more proactively as an attempt to offload some of the work being shared.\\

\subsection{Matthew Kottre}\\
\subsubsection{What technical information did you learn?}
\noindent I learned a lot about Django and the structure of Model/View frameworks. I also became much more proficient in database management from fixing a few problems that Django did not handle correctly.\\

\subsubsection{What non-technical information did you learn?}
\noindent I learned just how much writing and documentation is involved with projects. We spent far more time planning and writing documentation than we did coding. I've always been the kind of person who procrastinates on work, and that was not going to work well with a large project like this. As a result, my organization and time management skills significantly improved.\\ 

\subsubsection{What have you learned about project work?}
\noindent I learned how important planning and organization is. In order for each team member to effectively complete their part, they need to know how and what everyone else is doing as well.\\ 

\subsubsection{What have you learned about project management?}
\noindent I learned just how much writing and documentation is involved with projects. We spent far more time planning and writing documentation than we did coding. I also learned the importance of organization and time management.\\

\subsubsection{What have you learned about working in teams?}
\noindent I learned how important planning and organization is. In order for each team member to effectively complete their part, they need to know how and what everyone else is doing as well.\\

\subsubsection{If you could do it all over, what would you do differently?}
\noindent I would have liked to show more initiative throughout the project by volunteering to do certain tasks rather than waiting to be assigned left-overs. I have always been reactive, rather than proactive, so this project was a wake-up call telling me that I need to start planning ahead rather than waiting and hoping everything works out.\\

\subsection{Scott Waddington}\\

\subsubsection{What technical information did you learn?}
\noindent As a CS Systems Option, I don't get a lot of in-depth exposure to any one particular aspect of Computer Science. Being a member of this team and working on a web development project introduced me to a lot of aspects in web design that I wasn't familiar with, having only CS 290 under my belt as far as web programming experience is concerned. I also became somewhat familiar with Django, which I haven't used before. I don't claim to be an expert, but I can say with a degree of confidence that I could build a functional website in the future, with the experience that I gained from this project.\\

\subsubsection{What non-technical information did you learn?}
\noindent Documentation was huge throughout the capstone series. While we certainly could improve on some of our documentation, for the most part I think that we did a good job recording our progress and changes made to the project along the way. While we have all had to complete many assignments as an undergraduate, this one is unique because there was no clear guidelines for what needed to be done and how to implement it. Our client essentially gave us free reign on how we chose to design the application. This added a level of difficulty that we, as undergrads, rarely get to experience. Many assignments are strictly laid-out and the requirements are clearly communicated. This project gave us the chance to set out own schedule, come up with realistic expectations, and a plan to follow through, as well as the understanding that not everything is going to be implemented according to the first plan. The key is to adapt and to set realistic goals, and meet them.\\

\subsubsection{What have you learned about project work?}
\noindent As I said above, the most important thing we learned is that no plan survives the first point of contact. In essence, even the best plans for a project might change once implementation starts. There are things that occur that no one could foresee, and it's important to know how to adapt and overcome these challenges.\\

\subsubsection{What have you learned about project management?}
\noindent Communication is really important to a team, especially because we all have different schedules, unlike a job where everyone shows up at the same time and works. We all have other classes that we need to pass in order to graduate, and other obligations to tend to, so being able to communicate to one another was key to completing the project. At the beginning, we relied a lot on online communication and interaction, and as expo became nearer, we started meeting in-person more often in order to finish the project.\\

\subsubsection{What have you learned about working in teams?}
\noindent Again, I learned how much of an impact communication has on a team's success. It is important to have clear expectations of team members as well, and to ask for help when you need. Some people have more experience than others on certain aspects of the project, so consult those who know more than you. I didn't know anything about databases throughout the majority of the design and development of the project, so anytime I had to work with the database or create models in Django, I knew I had to ask one of my teammates for input.\\

\subsubsection{If you could do it all over, what would you do differently?}
\noindent Specific to this project, I think we would start working on CAS and AWS early. Both of these aspects took way longer to implement than we had expected, and it delayed some of the other features of our project until just before expo. I think we should have also worried less about the database, as we came to find out that Django had a lot of database management features build in, including the use of forms and models to represent objects in the database. I also think that if we had a stronger, more solid design in the beginning, it would have been easier to implement the project in the end.\\

\subsection{Mingwei Gao}\\
\subsubsection{What technical information did you learn?}
\noindent The most valuable technical information i learn from our project is how to design the database design and implementation in Django. Especially, I learn pretty much more when using what I learn from my advanced database building the data system. The six-stage of the design process of DBMS I learn from other class being translated into practice when I doing our project. (system, concept, logic, physics, implement, operation and maintenance)\\

\subsubsection{What non-technical information did you learn?}
\noindent The non-technical information that I learned should be the standard Latex writing and how to make the complete, well-formed documents. Since I am not a good writer, usually I won't encourage me to participate in any key writing part when I am in the large project. The weekly process record in our class helps me resolve this problem. Becasue I know what I did every week and what's going to do next time, I can manage my time and learning clearly. In other words, I know what I should write even more thinking after I review my weekly process. So, my learning management significantly improved. \\ 

\subsubsection{What have you learned about project work?}
\noindent Except for the deeper understanding of the technical problems arising from such a large-scale project, I learned how to work efficiently in the team including my clients and teammates. Also,  the project schedule management is the key role in our project. Project schedule management is the foundation and the key management stage to ensure the successful implementation of a project. \\ 

\subsubsection{What have you learned about project management?}
\noindent Communicate with your team, client, as to how we can improve our project management practices. Recording the week process is an important aspect of project management performed throughout the project. \\

\subsubsection{What have you learned about working in teams?}
\noindent I have learned the way how to work effectively in collaborative project teams. Every member needs providing guidance, support or assistance with each other as appropriate. All member should have high responsibility and efficiency to schedule, organize and coordinate projects with team spirit. \\

\subsubsection{If you could do it all over, what would you do differently?}
\noindent If I could do it all over again, I would complete the task a little earlier once I have received the team's requirement.  Even now I have the deep procrastination in my daily life, I have started making my daily work more structured by working with people who have a good life habit. Obviously, my teammates all have good habits and positive attitudes, their behavior was affecting my future work performance. \\