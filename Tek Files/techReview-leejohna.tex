\documentclass[letterpaper, 10pt, draftclsnofoot, onecolumn, IEEETran]{article}
\usepackage{fullpage} % changes the margin

\title{Student Letter of Recommendation Management Site}
\author{Johnathan Lee \\ {\tt CS461-001 Fall Term} \\ {\tt Professor Kirsten Winters}}
\date{November 9, 2018}

\begin{document}
\maketitle

\noindent
\subsection*{Abstract}

\paragraph{This document will explore some of the possible technologies that can be used in the implementation of our website. We will be taking a look at the different types of version control systems, methods of documentation, and methods of communication. I will provide some specific examples for each one and overview some of their traits.} 
\newpage


\section*{Introduction}
\paragraph{There are many different pieces of technology in the world today that can be used in the implementation of a website. Our group is focused on trying to make an Oregon State University website to manage student letter of recommendations from professors. This is a website designed to help professors maintain and follow up on student's requests for letters of recommendations from them. This will also help students streamline the process so they know what is expected of them. A couple of things that needs looking over for this website is what kind of version control system we want our project to have, our mode of communication between group members, and the type of documentation we want to be keeping.}
\section*{Version Control Systems}

\paragraph{The first technology to review are the version control systems. In a group setting, it's important to keep track of all the edits the project has gone through in case we need to revert to a previous edition. An example case where we need to do so is if a new addition completely broke the code and the website no longer worked. There are numerous VCS we could try using but the one that we're planning on using is Git. GitHub is a website that can hold and store developing projects in a repository and keeps each "version" of the project as people commit edits to it. It is one of the most commonly used version control and most, if not all of our team members are already familiar with it. It is considered a distributed version control system where there isn't exactly a centralized code base but rather different branches to pull code from. since it's required that we keep our stuff in a GitHub repository, we will most likely continue to use just this for our version control system.}

\paragraph{Another distributed version control system to consider is Bazaar. Bazaar is similar to Git in that it also stores different versions of your projects. One of the main differences is the number of options that are given over the setup. Bazaar offers complete control for the user to optimize the setup to the project \cite{VCS}. It can set up many different types of control systems as well to fully customize and fit to the user's situation. Because of its adaptability, it can fit to any scenario and to any project that you need. It also offers embedding so it can be added to existing projects \cite{VCS}. It is a user-friendly website that helps collaborators easily work together over a project.}

\paragraph{A third version control system that we could possibly consider using or at least to review is Mercurial. Mercurial is known for its speed which the developers built with performance in mind \cite{VCS}. It is another distributed version control system that offers fast and easy version control. It offers a tutorial on its main page and a link to available extensions to enhance its usability. It also offers an example of how to quickly use Mercurial to create a project and commit it via command-line. The main page is a great place to get a quick start on the system if we ever chose to use it. Like the other free version control systems, this one is also very user-friendly and offers an efficient workflow. One thing to take in mind though is that this is generally used for bigger projects \cite{Merc}. It can handle smaller projects just fine, but this is the more popular option if you ever need a VCS for some big project.}

\section*{Documentation}
\paragraph{We will need a way of keeping track of our progress on the project. There are a number of ways to keep documentation of communicating what we've done. The first documentation technology we'll explore is LaTeX. LaTeX is a typesetting system that is used for most scientific documents. It provides immense control to the user to be able to add, change, or modify whatever they need or want to their document. Since we're required to use LaTeX, this is what we will be using to record what we've done with the project. After typing our documentation in LaTeX, we can convert the writing to a PDF for viewing. One of the many benefits of using LaTeX over most other writing programs is that it usually looks much nicer because of the customization options. Overleaf is one of the many ways of writing in LaTeX. It provides a streamlined experience of using LaTeX writing and also offers a way to convert the given tex file into a PDF. It also allows you to store all of your LaTeX projects you've worked on with a free account.}

\paragraph{A second method of documentation that we could review is Microsoft Word. Microsoft Word is a simple to use program that can save whatever you type in a document file. It is a common program that can be found on most computers with an easy to use user interface. It allows for fast and easy font or style changes that can be made on the fly. This is something we could consider using just to jot down notes to transfer over to another program later. However, the third example I will be talking about is more practical for this.}

\paragraph{A third method of documentation is Google Docs. This is a piece of software technology that's integrated with Google which our OSU emails are associated with. This piece of technology is special because we can use this to share documents with each other. This is one way of getting each other to write on the same document because it allows people to work on the same document simultaneously. It offers fast and easy to use sharing that can help us get on the same page. Since we've already shared our emails with each other, we can easily collaborate on a Google Doc with each other. It can be a way of getting an idea started together as a group and can be easily transferred over to LaTeX.}

\section*{Communication}
\paragraph{As a group project, it is important to keep in touch with other group members. To immediately let others know what you're doing and how far along you are on a process is invaluable. The current method of communication we're using is Gmail through the use of our OSU emails. Tagging everyone in the group with the email lets everyone know what's going on and allows them to reply to the group email. However, compared to instant messaging apps it's not as efficient or fast for group messaging. It is a more formal way of getting in touch with group members and proposing an idea. The fact that the mail needs to be composed, sent, then opened to read takes a lot of time to get an idea across, and takes even longer when a response is needed.}

\paragraph{One other way of communicating is instant text messaging through the use of phones. Sharing each others' phone numbers and creating a group text with the group members in it allows one to get an alert immediately when someone has messaged the group. Phones are usually always at our side and thus provide instant access if the need arises. One potential drawback of this is that it's usually not very cross platform friendly. There are oftentimes glitches that occur when Android and iPhones text each other. Still, it's very convenient to be able to send and receive a text instantly which allows ideas to communicate freely.}

\paragraph{There are many apps that allow groups to communicate with each other. One such app is Discord, which can be downloaded for free on mobile or on PC, or used on the web browser. It's an app that allows text and voice chat, screen share, and even video sharing. It has become a popular messaging app for many other groups and can handle many different servers for different groups you're associated with. The user interface is not very hard to handle and individual chat channels can be muted. Voice chat is useful if the group members are not able to meet up on campus. Its screen share is also very useful if you're trying to show something without being in person. It has many different functions for a messaging app. Most of these functions can be ignored for just the basics of text based communication or voice chat.}

\section*{Conclusion}
\paragraph{It is important to make sure that all team members are on the same page using the same technology. It makes it easier to compare resources with far less hiccups in the process. I have highlighted some of the options that we could use. However, there are many more options that I have not explored that could be utilized as well. There are many options for version control systems, methods of documentation, and methods of communication that are all free to use. We are not limited to just the examples that I have found. Even though it's most likely we are going to stick with what we've found already, we could still decide to change what we've been doing if we find a more effective option for our project moving forward.}

\newpage
\begin{thebibliography}{}
\bibitem{VCS}
"Version Control Systems" [Online]. Available: https://www.smashingmagazine.com/2008/09/the-top-7-open-source-version-control-systems/
\bibitem{Merc}
"Mercurial" [Online]. Available: https://www.mercurial-scm.org/

\end{thebibliography}

\end{document}
