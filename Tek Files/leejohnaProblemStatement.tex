\documentclass[letterpaper, 10pt, draftclsnofoot, onecolumn, IEEETran]{article}
\usepackage{fullpage} % changes the margin
\title{Student Letter of Recommendation Management Site}
\author{Johnathan Lee \\ {\tt CS461-001 Fall Term} \\ {\tt Professor Kirsten Winters}}
\date{October 11, 2018}

\begin{document}
\maketitle

\noindent
\subsection*{Abstract}
This paper is designed to get an idea of what the project will be about and what it aims to accomplish. The team seeks to create a website that is designed to streamline the process of requesting a recommendation letter from a professor. It will be a part of Oregon State University's websites and will be accessible to Oregon State students and faculty. It will hold numerous features and style designs that the team has yet to decide on. The main points of this project will be tracking the recommendation letter process, moving part of the process online and streamlining it for ease of access, and documenting and holding important information needed for the process. In short, it will be managing the entire process for both the professor and the student involved.
\newpage
\section*{Problem Statement}
The problem that our team is focusing on is the fact that when students ask for a recommendation letter from a professor, it requires a lot of work from both the student and the professor. The professor needs to know who they're writing this recommendation letter out to, and keep track of this student. The student needs to provide information to the professor so that they know who they're writing out to and why they should even get a recommendation letter from them. The professor sometimes requires the student to procure specific documents that can be time consuming and hard to keep track of. Unless the student particularly stands out from the class, the professor won't know if they deserve a recommendation letter. We need something that can help keep track of all these things so it isn't so cumbersome to even simply ask for recommendation letters.

\section*{Solution}
The solution to this problem is designing a website that will store all of this information and streamline the recommendation letter process. Documents can be sent over through this site and will save and keep track of this information. Some important information that needs to be seen by the professor is the student's record of taking their class and the grade they received while they were in the class. This is an important documentation that they need to verify that they indeed have been in their class and did well. I am sure that there are other pieces of important information that we need to keep track of but I do not know yet since we have not started or heard back from the client. So for now, we have a very rough idea of what this website should look like.


\section*{What it should have (Performance)}
 We can have student accounts and teacher accounts so that a student can request a recommendation letter from a professor and it will send the professor some sort of notification that they can manage the task on their own time. The student can send over the important information that the professor needs to fill out the recommendation letter and when the professor finishes the recommendation letter, they can verify that they have already processed the recommendation letter and close the request. A lot of our time will be spent on the CSS of the website and styling it properly. Since this website will be affiliated with Oregon State University, it needs to look like it's part of Oregon State University and be on their servers. We will know if it's done when we can access the website from one of our computers and make a new account and login and contact a professor account with a recommendation letter request. Another important thing will be is that it will look finished with Oregon State University colorings and styles (i.e. an orange and black finish). This entire process should be very streamlined so there will be many prompts to fill out so that everything that's required can be sent over right away. This can work out similarly to emails where you have an inbox of requests from students for recommendation letters while a student has an inbox for information requests from professors. It's important to note that some of these things can be cut out if they take too long to design and implement into the website. There are a lot of ideas that can be thrown out that would be cool and functional to have, but wouldn't be necessary to implement live. Quality of life changes can be pushed out after it's actually fulfilling it's original purpose of streamlining a recommendation letter request process.

\end{document}
